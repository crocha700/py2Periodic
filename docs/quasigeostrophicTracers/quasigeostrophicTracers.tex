\documentclass[12pt, oneside]{article}

\usepackage{float}
\usepackage{lineno}
\usepackage{color, amssymb, amsmath, amsthm, verbatim, wasysym}
\usepackage{natbib}
\usepackage{epsfig}
\usepackage[mathscr]{eucal}
\usepackage{mathrsfs}
\usepackage{appendix}
\raggedbottom
\usepackage[left=0.8in,right=0.8in,top=0.8in,bottom=0.9in,centering]{geometry}      

% Uncomment to show references.
%\usepackage[notcite,notref]{showkeys}

% To make really wide hats that cover everything:
\usepackage{scalerel}
\usepackage{stackengine}
\setstackEOL{\#}
\stackMath
\def\hatgap{1pt}
\def\subdown{-0.2pt}
\newcommand\reallywidehat[2][]{%
\renewcommand\stackalignment{l}%
\stackon[\hatgap]{#2}{%
\stretchto{%
    \scalerel*[\widthof{$#2$}]{\kern-.6pt\bigwedge\kern-.6pt}%
    {\rule[-\textheight/2]{1ex}{\textheight}}%WIDTH-LIMITED BIG WEDGE
}{0.6ex}% THIS SQUEEZES THE WEDGE TO 0.5ex HEIGHT
_{\smash{\belowbaseline[\subdown]{\scriptstyle#1}}}%
}}

% For a 'strut' that creates space above underbraces
\newcommand*\mystrut[1]{\vrule width0pt height0pt depth#1\relax}

% Punctuation
\newcommand{\com}{\, ,}
\newcommand{\per}{\, .}

% A nice 'definition'
\newcommand{\defn}{\ensuremath{\stackrel{\mathrm{def}}{=}}}

% Use \bar to over line solo symbols
\newcommand{\av}[1]{\left \langle{#1} \right \rangle}
\newcommand{\avbg}[1]{\overline{#1}}
\newcommand{\avbgg}[1]{\overline{#1}}
\newcommand{\hav}[1]{\widehat{#1}}

% Begin and end equations
\newcommand{\beq}{\begin{equation}}
\newcommand{\eeq}{\end{equation}}

% Vector calculus operators
\newcommand{\p}{\partial}
\newcommand{\bnabla}{\boldsymbol \nabla}
\newcommand{\pnabla}{\boldsymbol \nabla_{\! \! \perp}}
\newcommand{\hnabla}{\bnabla_{\! \! h}}
\newcommand{\bnablad}{\bnabla_{\! \! \alpha}}
\newcommand{\bcdot}{\boldsymbol \cdot}
\newcommand{\hlap}{\triangle_h}
\newcommand{\lap}{\triangle}
\newcommand{\grad}{\bnabla}
\newcommand{\curl}{\bnabla \!\times\!}
\newcommand{\diver}{\bnabla \bcdot }
\newcommand{\cross}{\times}

% Bold symbolds
\newcommand{\bu}{\boldsymbol u}
\newcommand{\buh}{\boldsymbol u_h}
\newcommand{\bx}{\boldsymbol x}
\newcommand{\ba}{\boldsymbol{a}}
\newcommand{\bk}{\boldsymbol{k}}
\newcommand{\bh}{\boldsymbol{h}}
\newcommand{\bm}{\boldsymbol{m}}
\newcommand{\bn}{\boldsymbol{\hat n}}
\newcommand{\bxh}{\hspace{0.1em} \boldsymbol{\hat x}}
\newcommand{\byh}{\hspace{0.1em}\boldsymbol{\hat y}}
\newcommand{\bzh}{\hspace{0.1em}\boldsymbol{\hat z}}
\newcommand{\bnh}{\hspace{0.1em}\boldsymbol{\hat n}}
\newcommand{\bomega}{\boldsymbol \omega}
\newcommand{\bOmega}{\boldsymbol \Omega}
\newcommand{\bxi}{\ensuremath {\boldsymbol {\xi}}}
\newcommand{\bXi}{\ensuremath {\boldsymbol {\Xi}}}
\newcommand{\bU}{\boldsymbol{U}}
\newcommand{\bX}{\boldsymbol{X}}

% Greek abbrevs
\newcommand{\ep}{\epsilon}
\newcommand{\om}{\omega}
\newcommand{\kap}{\kappa}

% Roman characters
\newcommand{\ee}{\mathrm{e}}
\newcommand{\ii}{\mathrm{i}}
\newcommand{\cc}{\mathrm{cc}}
\newcommand{\dd}{{\rm d}}
\newcommand{\id}{{\, \rm d}}
\newcommand{\DD}{{\rm D}}
\newcommand{\J}{\mathrm{J}}
\renewcommand{\L}{\mathrm{L}}

% Non-dimensional numbers 
\newcommand{\Ri}{Ri}
\newcommand{\Ro}{Ro}
\newcommand{\Bu}{Bu}
\newcommand{\Pe}{Pe}

% Material derivative
\newcommand{\Dt}[1]{\mathrm{D}_t #1}

% Small in-line fractions
\newcommand{\half}{\tfrac{1}{2}}

% Bold 'F' for 'forcing'
\newcommand{\bff}{\boldsymbol{F}}
\newcommand{\fh}{\breve f}

% Dissipation operator
\newcommand{\friction}{\mathrm{F}}
\newcommand{\mixing}{\mathrm{M}}

\newcommand{\mode}{\phi}
\newcommand{\q}{\tilde q}
\newcommand{\qpsi}{\tilde \psi}

\begin{document}

\title{\vspace{-4ex} Quasi-geostrophic flow with tracers}
\author{Greg}
\date{} \maketitle \vspace{-4ex}

\section{Preliminaries}

More or less according to Vallis (2006), two-layer quasi-geostrophic flow is governed by the equations
\begin{align}
\q_{1t} + \J \big ( \qpsi_1, \q_1 \big ) &= \friction \big ( \hlap \qpsi_1 \big ) \\
\q_{2t} + \J \big ( \qpsi_2, \q_2 \big ) &= -r \hlap \qpsi_2 + \friction  \big( \hlap \qpsi_2 \big ) \com 
\end{align}
where the potential vorticities $Q_i$ are defined by
\begin{align}
\q_{1} &= \hlap \qpsi_1 + F_1 \big ( \qpsi_1 - \qpsi_2 \big ) + \beta y \com \\
\q_2 &= \hlap \qpsi_2 - F_2 \big ( \qpsi_1 - \qpsi_2 \big ) + \beta y \com
\end{align}
and the Burger numbers $F_1$ and $F_2$ are defined by $F_i = f_0^2 / g' H_i$, where $H_i$ is the height of layer $i$ and $g'$ is the reduced gravity. The number of parameters in the problem are reduced if the deformation `radius' $R$ and layer depth ratio $\delta$ are defined through
\beq  
R^2 \defn \frac{g' H_1 H_2}{f_0^2 \left ( H_1 + H_2 \right )} \com \qquad \text{and} \qquad \delta \defn \frac{H_1}{H_2} \com
\eeq
so that 
\beq
F_1 = \frac{R^{-2}}{1 + \delta} \com \qquad \text{and} \qquad F_2 = \frac{\delta R^{-2}}{1 + \delta} \per
\eeq
We decompose $\qpsi_i$ into $\qpsi_i = - U_i y + \psi_i$, so that
\begin{align}
\q_1 &= \underbrace{\mystrut{1.4ex}  \beta y - F_1 \big ( U_1 - U_2 \big )}_{\defn Q_1} + \underbrace{\mystrut{1.4ex} \hlap \psi_1 + F_1 ( \psi_1 - \psi_2 )}_{\defn q_1} \com \\
\q_2 &= \underbrace{\mystrut{1.4ex}  \beta y + F_2 \big ( U_1 - U_2 \big )}_{\defn Q_2} + \underbrace{\mystrut{1.4ex} \hlap \psi_2 - F_2 ( \psi_1 - \psi_2 )}_{\defn q_2} \com
\end{align}
The potential vorticity conservation equations become
\begin{align}
q_{1t} + \J \left( \psi_1, q_1 \right ) + U_1 q_{1x} + \psi_{1x} Q_{1y} &= \friction \left ( \hlap \psi_1 \right) \com \\
 q_{2t} + \J \left( \psi_2, q_2 \right ) + U_2 q_{2x} + \psi_{2x} Q_{2y} &= - r \hlap \psi_2 + \friction \left ( \hlap \psi_2 \right) \per
\end{align}

\section{Linear stability analysis}

\bibliographystyle{jfm}
\bibliography{refs}

\end{document}